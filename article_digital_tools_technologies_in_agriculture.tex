% Options for packages loaded elsewhere
\PassOptionsToPackage{unicode}{hyperref}
\PassOptionsToPackage{hyphens}{url}
%
\documentclass[
]{article}
\usepackage{amsmath,amssymb}
\usepackage{iftex}
\ifPDFTeX
  \usepackage[T1]{fontenc}
  \usepackage[utf8]{inputenc}
  \usepackage{textcomp} % provide euro and other symbols
\else % if luatex or xetex
  \usepackage{unicode-math} % this also loads fontspec
  \defaultfontfeatures{Scale=MatchLowercase}
  \defaultfontfeatures[\rmfamily]{Ligatures=TeX,Scale=1}
\fi
\usepackage{lmodern}
\ifPDFTeX\else
  % xetex/luatex font selection
    \setmainfont[]{Times New Roman}
\fi
% Use upquote if available, for straight quotes in verbatim environments
\IfFileExists{upquote.sty}{\usepackage{upquote}}{}
\IfFileExists{microtype.sty}{% use microtype if available
  \usepackage[]{microtype}
  \UseMicrotypeSet[protrusion]{basicmath} % disable protrusion for tt fonts
}{}
\makeatletter
\@ifundefined{KOMAClassName}{% if non-KOMA class
  \IfFileExists{parskip.sty}{%
    \usepackage{parskip}
  }{% else
    \setlength{\parindent}{0pt}
    \setlength{\parskip}{6pt plus 2pt minus 1pt}}
}{% if KOMA class
  \KOMAoptions{parskip=half}}
\makeatother
\usepackage{xcolor}
\usepackage[top=0.85in,left=1in,footskip=0.75in,marginparwidth=2in]{geometry}
\usepackage{longtable,booktabs,array}
\usepackage{calc} % for calculating minipage widths
% Correct order of tables after \paragraph or \subparagraph
\usepackage{etoolbox}
\makeatletter
\patchcmd\longtable{\par}{\if@noskipsec\mbox{}\fi\par}{}{}
\makeatother
% Allow footnotes in longtable head/foot
\IfFileExists{footnotehyper.sty}{\usepackage{footnotehyper}}{\usepackage{footnote}}
\makesavenoteenv{longtable}
\usepackage{graphicx}
\makeatletter
\def\maxwidth{\ifdim\Gin@nat@width>\linewidth\linewidth\else\Gin@nat@width\fi}
\def\maxheight{\ifdim\Gin@nat@height>\textheight\textheight\else\Gin@nat@height\fi}
\makeatother
% Scale images if necessary, so that they will not overflow the page
% margins by default, and it is still possible to overwrite the defaults
% using explicit options in \includegraphics[width, height, ...]{}
\setkeys{Gin}{width=\maxwidth,height=\maxheight,keepaspectratio}
% Set default figure placement to htbp
\makeatletter
\def\fps@figure{htbp}
\makeatother
\setlength{\emergencystretch}{3em} % prevent overfull lines
\providecommand{\tightlist}{%
  \setlength{\itemsep}{0pt}\setlength{\parskip}{0pt}}
\setcounter{secnumdepth}{-\maxdimen} % remove section numbering

% % use Unicode characters - try changing the option if you run into troubles with special characters (e.g. umlauts)
% \usepackage[utf8]{inputenc}
\usepackage{fontspec}

% hyperref makes references clicky. use \url{www.example.com} or \href{www.example.com}{description} to add a clicky url
\usepackage{nameref,hyperref}

% line numbers
\usepackage[right]{lineno}

% use adjustwidth environment to exceed text width (see examples in text)
\usepackage{changepage}

% adjust caption style
\usepackage[aboveskip=1pt,labelfont=bf,labelsep=period,singlelinecheck=off]{caption}

% use \textcolor{color}{text} for colored text (e.g. highlight to-do areas)
\usepackage{color}

% define custom colors (this one is for figure captions)
\definecolor{Gray}{gray}{.25}

% this is required to include graphics
\usepackage{graphicx}

% use if you want to put caption to the side of the figure - see example in text
\usepackage{sidecap}

\newcommand{\blandscape}{\begin{landscape}}
\newcommand{\elandscape}{\end{landscape}}

\usepackage{bm} % for supporting bold math fonts
\usepackage{siunitx}
\usepackage{moreverb}
\usepackage{booktabs}
\usepackage{longtable}
\usepackage{array}
\usepackage{multirow}

% use for have text wrap around figures
\usepackage{wrapfig}
\usepackage[pscoord]{eso-pic}
\usepackage[fulladjust]{marginnote}
\reversemarginpar

\usepackage{float}
\usepackage{colortbl}
\usepackage{pdflscape}
\usepackage{tabu}
\usepackage{threeparttable}
\usepackage{threeparttablex}
\usepackage[normalem]{ulem}
\usepackage{makecell}
\usepackage{xcolor}
\usepackage{tikz} % required for image opacity change
\usepackage[absolute,overlay]{textpos} % for text formatting
\usepackage{chemfig}

\sisetup{per-mode=symbol}

% properly format quotes, hyphenations, dates, etc.
\usepackage{csquotes}
\usepackage[american]{babel}

% Citation and referencing
\usepackage[style=apa,backend=biber,sorting=nyt]{biblatex}
\DeclareLanguageMapping{american}{american-apa}


% Added by CII
% \usepackage[format=hang,labelfont=bf,margin=0.5cm,justification=centering]{caption} # don't use bf
\usepackage[format=hang,margin=0.5cm,justification=centering]{caption}
\captionsetup{font=small,width=0.9\linewidth,labelfont=small,textfont={small}}
% End of CII addition

\usepackage{subcaption}
% \newcommand{\subfloat}[2][need a sub-caption]{\subcaptionbox{#1}{#2}}

% \captionsetup[sub]{font=footnotesize}
\captionsetup[subfigure]{font=small,labelfont=small,textfont=small}
\ifLuaTeX
  \usepackage{selnolig}  % disable illegal ligatures
\fi
\usepackage[]{biblatex}
\addbibresource{bibfile.bib}
\usepackage{bookmark}
\IfFileExists{xurl.sty}{\usepackage{xurl}}{} % add URL line breaks if available
\urlstyle{same}
\hypersetup{
  pdftitle={Digital Tools and Technologies in Agriculture: Extensibility, Institutional Learning, and Prospects for Nepal},
  hidelinks,
  pdfcreator={LaTeX via pandoc}}

\title{Digital Tools and Technologies in Agriculture: Extensibility, Institutional Learning, and Prospects for Nepal}
\author{Deependra Dhakal\footnote{Agriculture and Forestry University, Chitwan, Nepal; Email: \url{ddhakal@afu.edu.np}}, Hari Paneru\footnote{Agriculture Knowledge Center, Kanchanpur, MoLMAC, Sudurpaschim Province, Nepal}}
\date{2025-12-25}

\begin{document}
\maketitle
\begin{abstract}
Digital technologies are increasingly reshaping agricultural systems by enabling data-driven decision making, precision management, and improved governance across value chains. In Nepal, however, the adoption of agricultural technologies has historically been episodic -- characterized by pilot-driven introduction, partial diffusion, and eventual abandonment -- often due to weak institutional embeddings and limited capacity for continued improvement. This article examines digital technologies not as discrete tools, but as extensible infrastructures whose value lie in cumulative learning, modularity, and adaptability. Through analytical discussion and applied examples, current paper demonstrates how digital tools can address long-standing challenges in agricultural statistics and planning. District-stratified crop-cut surveys augmented with remote sensing data are proposed for improving crop yield estimation by reducing subjectivity and enhancing spatial completeness. Similarly, crop suitability mapping using public-domain geospatial data illustrates how transparent, scalable decision-support systems can guide land-use planning and mechanization strategies.
\end{abstract}

\textbf{Keywords}: Digital map, Data pipeline, Modularity, Suitability, Yield estimation

\section{Scope and Domains of Digital Technologies in Agriculture}\label{scope-and-domains-of-digital-technologies-in-agriculture}

In the agricultural context, digital technologies refer to hardware--software assemblages that enable systematic data generation, transmission, processing, and decision support across farming operations. Their relevance spans the entire agricultural value chain -- from land preparation and crop management to marketing, insurance and policy governance. In practical terms, these technologies encompass:

\begin{itemize}
\tightlist
\item
  Sensing systems (remote sensing platforms, UAVs, soil and weather sensors),
\item
  Positioning and navigation tools (GPS-enabled machinery and field mapping systems),
\item
  Data platforms (farm management systems, geographic information systems),
\item
  Communication interfaces (mobile applications, dashboards, alert systems),
\item
  Analytical engines (statistical models, simulation tools, machine learning algorithms).
\end{itemize}

A shared feature of all these systems/components that constitute the digital ecosystem is the reliance on data as productive asset. Unlike conventional mechanization, which primarily substitutes labor with power, digital systems amplify cognition---allowing farmers, planners, and institutions to observe patterns, anticipate risks, and coordinate actions at scales that were previously impractical.

Unlike conventional mechanization, digital tools are not single-purpose artifacts. They operate as platform technologies, capable of incremental upgrades, modular replacement, and cross-domain integration. A drone platform, for instance, can be successively repurposed -- from acreage estimation to crop stress detection to insurance verification -- without replacing the underlying flight enabling hardware and it's components. This platform character fundamentally distinguishes digital technologies from earlier generations of agricultural interventions that were often rigid, capital-intensive, and context-specific.

\section{From Episodic Adoption to Cumulative Digital Transformation}\label{from-episodic-adoption-to-cumulative-digital-transformation}

The integration of digital tools into agriculture is no longer a question of novelty but of institutional maturity and cumulative learning. Across the globe, agriculture is transitioning from systems governed primarily by experience and seasonal intuition, orchestrated by labor-intensive systems toward data-mediated, sensor-driven, and algorithmically assisted production systems. This transition is not defined merely by the introduction of new tools, but by their continued refinement, contextual adaptation, and systemic embedding within production, governance, and market ecosystems.

In Nepal, however, the historical trajectory of agricultural technology adoption has often followed a discontinuous path: trial, partial diffusion, abandonment, and eventual replacement by another externally promoted innovation. This pattern has affected not only mechanical tools but also irrigation schemes, seed systems, and extension models. The present article argues that digital technologies -- if understood as extensible infrastructures rather than fixed interventions -- offer an opportunity to break this cycle. Their success, however, relies upon institutional readiness, domain-specific digital literacy, and robust system design.

Although, as of recent Agriculture Information and Training Center (AITC), Nepal is actively deploying digital platforms such as the Kishan (Farmers') Portal, digital call center services, and piloting digital solutions at the local level, as is the case for most digital tools, penetration among Nepalese farming community is little. Farmers registration through the use if Kishan portal aims to bridge the entry point and all-round service delivery via digital route to farmers.

\section{Institutional and Policy Landscape for Digital Agriculture in Nepal}\label{institutional-and-policy-landscape-for-digital-agriculture-in-nepal}

Nepal's policy environment for information and communication technology (ICT) has evolved steadily over the past three decades. Early liberalization through the National Communication Policy (1992) and the Telecommunication Act (1996) laid the foundation for widespread connectivity, while subsequent IT policies (2000, 2010) framed digital tools as catalysts for socio-economic transformation rather than mere communication utilities.

More recent initiatives -- including the Digital Nepal Framework (2019), the ICT Policy (2015), cybersecurity bylaws (2020), the National Cyber Security Center, and the National Cyber Security Policy (2023) -- signal a growing recognition of digital infrastructure as critical national infrastructure. Together, these frameworks emphasize resilience, interoperability, and global integration.

From an agricultural standpoint, this policy ecosystem is sufficiently flexible to permit experimentation, pilot programs, and donor-supported digital initiatives. What remains underdeveloped, however, is the institutional capacity to absorb these technologies into routine practice, to iteratively improve them, and to scale successful models beyond project lifecycles. The challenge, therefore, is not regulatory absence, but institutional depth and continuity.

\section{Institutional Fragility and Lessons from Infrastructure Failures}\label{institutional-fragility-and-lessons-from-infrastructure-failures}

Nepal's development history provides several instructive examples of infrastructure projects that delivered far less public value than originally anticipated. Large-scale irrigation schemes such as the Sikta Irrigation Project and the Bheri--Babai Diversion Project, despite substantial financial investment, have struggled with underutilization, delayed commissioning, maintenance bottlenecks, and limited alignment with local cropping systems. Similar critiques have emerged in relation to rural road expansion, where physical connectivity improved rapidly, but agricultural commercialization and market integration lagged behind.

These outcomes are not merely the result of funding constraints or administrative inefficiencies. They point to a deeper structural issue: systems designed for construction rather than for operation, learning, and adaptation. As emphasized in classical systems engineering and organizational theory \autocite{simon2019sciences,meadows2008thinking}, robust systems are those that incorporate feedback loops, modular components, adaptive capacity, and institutional learning mechanisms. Systems that lack these properties may function under initial assumptions, but tend to degrade when conditions change.

This distinction is particularly relevant for agriculture, where environmental variability, market volatility, and behavioral responses are intrinsic features rather than anomalies. Traditional agricultural infrastructure in Nepal has often been conceived as static -- once built, expected to function indefinitely -- while agricultural realities are inherently dynamic.

Digital infrastructures offer a partial corrective. By design, they are modular, upgradable, and reversible. Software systems can be patched, sensors recalibrated, data models refined, and workflows reorganized without dismantling the entire system. However, this technical flexibility does not automatically translate into agricultural impact. Effective deployment requires institutional maneuverability, which is largely determined by human capacity---specifically, understanding of data structures, model assumptions, and system limitations. Consequently, domain-specific digital literacy emerges as a foundational requirement for durable digital agriculture.

\section{Relevance of Digital Technologies to Nepalese Agriculture}\label{relevance-of-digital-technologies-to-nepalese-agriculture}

Digital technologies hold particular relevance for Nepal's agricultural context because they align well with its structural constraints: small farm sizes, ecological heterogeneity, limited capital intensity, and decentralized governance.

\subsection{Mechanization, precision, and smallholder compatibility}\label{mechanization-precision-and-smallholder-compatibility}

Nepal's agriculture is dominated by smallholder farms operating under fragmented landholdings and diverse agro-ecological conditions. While heavy mechanization faces clear physical and economic limits, scale-appropriate tools---such as mini-tillers---have already demonstrated productivity gains even in hill systems \autocite{paudel2019scale}. When combined with basic digital augmentation (GPS guidance, plot-level mapping, activity logging), such tools become gateways to precision agriculture.

Importantly, digital augmentation does not require full mechanization. Even manual or semi-mechanized operations can benefit from spatial referencing, decision-support tools, and temporal record-keeping, enabling farmers to manage variability. A recent venture in field application of pesticide/fertilizer in sugarcane, a naturally tall-growing plant posing limits in spraying chemical preparations, using drone assisted spraying has received postive reviews from growers in Kanchanpur, Nepal \autocite{DroneUseKanchanpur2025}.

\subsection{Digital governance, interoperability, and service delivery}\label{digital-governance-interoperability-and-service-delivery}

Digital infrastructures also reshape how agricultural services are governed and delivered. Interoperable systems allow farm-level data to flow upward into municipal planning, provincial aggregation, and national reporting without repeated manual consolidation.

E-governance platforms -- already in use for G2C (a prominent example being the ``Nagarik App''), G2G (used mostly in the context of foreign employment and post-disaster rehabilitation aid), and G2E (web-based internal communication and personnel database access, etc.) interactions -- introduce standardized validation rules, reduce discretionary ambiguity, lower service-delivery latency, and create traceable decision pathways. In agriculture, this translates into faster subsidy disbursement, clearer beneficiary targeting, resolution of redundancy in program implementation, and real-time monitoring of program implementation. Crucially, it also enables adaptive governance, where emerging risks---such as pest outbreaks or climatic anomalies---can be addressed during the season rather than after losses have occurred.

\subsection{Continuous monitoring, forecasting, and risk management}\label{continuous-monitoring-forecasting-and-risk-management}

Digital crop monitoring systems enable continuous observation of crop growth, stress, and variability. When integrated with historical production records and market information, these systems support more rational decisions regarding crop choice, storage, marketing, and insurance.

For agricultural insurance, for instance, an enterprise or cooperative can use time-series yield estimates derived from remote sensing to:

\begin{itemize}
\tightlist
\item
  Assess historical yield volatility,
\item
  Classify risk zones spatially,
\item
  Select insurance products aligned with actual exposure,
\item
  Reduce moral hazard and dispute frequency.
\end{itemize}

Such data-driven underwriting is already common in index-based insurance schemes globally and is directly applicable to Nepal's emerging agri-insurance sector.

\section{Cases of Digital Technology Extension in Nepalese Agriculture}\label{cases-of-digital-technology-extension-in-nepalese-agriculture}

\subsection{Improving information reliability in yield estimation}\label{improving-information-reliability-in-yield-estimation}

Estimation of crop yield constitutes a major function of national agricultural governing body. Estimation combined with aggregation and reporting of nation/sub-national/local level production and yield data serves as a quantitative evidence to inform decision-making for agricultural policy and programs.

National yield estimation traditionally relies on crop-cutting experiments (CCE) and farmer recall surveys. While CCEs are considered statistically rigorous, their reliability is contingent upon:

\begin{itemize}
\tightlist
\item
  Sampling frame completeness,
\item
  Randomization integrity,
\item
  Enumerator competence,
\item
  Temporal alignment with harvest.
\end{itemize}

In Nepal, absence of uniform sampling design and standardized protocols has led to discrepancies in reported yield figures across government agencies and between governmental and non-governmental organizations, a phenomenon widely documented in developing-country contexts. For example, a national media outlet underlined the discripancy in per capita consumption statistic of milk, meat and egg in 2017 report produced by two Nepal Rastra Bank and Ministry of Livestock Development \autocite{DataDiscrepancyCreates2017}.

When yield is inferred from secondary information -- expert opinion, market arrivals, or anecdotal assessments -- the epistemic basis becomes even weaker \autocite{baumeister2007psychology}. The grounds for inference are often undocumented, rendering estimates non-reproducible and disputed.

Digital tools provide an opportunity to constrain subjectivity and behavioral bias in yield estimation. Remote sensing, UAV imagery, and automated data pipelines introduce observer-independent measurements that can be repeatedly validated against ground truth. Rather than replacing crop-cut surveys, digital systems can narrow uncertainty bounds, provide spatial completeness, and statistical inference enforces internal consistency. This eventually alleviates burden of estimation and judgement for human to focus on interpretation. Such hybrid systems are increasingly recognized as best practice in agricultural statistics.

\subsubsection{A statistical framework for national crop yield estimation using district-stratified crop-cut surveys}\label{a-statistical-framework-for-national-crop-yield-estimation-using-district-stratified-crop-cut-surveys}

Crop yield estimation in Nepal is commonly implemented using a stratified multi-stage sampling design, where districts constitute the primary strata. This choice is operationally pragmatic, as districts represent the lowest administrative unit at which routine agricultural statistics are aggregated and reported, and they often capture broad agro-ecological and management variability.

Let:

\begin{itemize}
\tightlist
\item
  \(s = 1, 2, \dots, S\) index the districts (strata),
\item
  \(A_s\) denote the total cultivated area (ha) of a given crop in district (\(s\)),
\item
  \(n_s\) denote the number of crop-cut plots sampled in district (\(s\)),
\item
  \(y_{si}\) denote the observed yield (e.g., \(kg~ha^{-1}\)), standardized to a reference moisture level) from the (\(i^{th}\)) crop-cut plot in district (\(s\)).
\end{itemize}

The district-level mean yield estimator is:

\[
\hat{\bar{Y}}*s = \frac{1}{n_s} \sum*{i=1}^{n_s} y_{si}
\]

The national mean yield is obtained as an area-weighted aggregation of district-level estimates:

\[
\hat{\bar{Y}} = \frac{\sum_{s=1}^{S} A_s \hat{\bar{Y}}*s}{\sum*{s=1}^{S} A_s}
\]

Accordingly, the estimated national crop production (\(\hat{P}\)) is expressed as:

\[
\hat{P} = \sum_{s=1}^{S} A_s \hat{\bar{Y}}_s
\]

This estimator assigns greater influence to districts with larger cropped areas while preserving inter-district yield heterogeneity. Its statistical validity relies on the representativeness of crop-cut plots within each district, strict adherence to randomization protocols, and consistency in harvest and moisture-adjustment procedures.

Under Nepal's federal governance structure, this district-stratified framework can be naturally extended to incorporate municipalities and wards as successive sampling stages, without altering the unbiasedness of the estimator.

\subsubsection{Arial imaging systems as survey multipliers}\label{arial-imaging-systems-as-survey-multipliers}

Although direct assessment of yield is always possible, which is often cost-prohibitive and as a consequence is extremely information-sparse \autocite[Pp. 419 of][]{gomez1984statistical}, investigators can now refer to more revealing components that are scientifically shown be correlated with the yield. An example approach is using the drone based imaging that collects information on field attributes such as crop density, height, weed density, etc. in terms of spectral emissions, which ultimately can be used to calculate net productivity of the region of interest. This can be used, after validation through on-field crop cut survey, for comparison across wide area. Precision agriculture applications typically require spatial resolutions of 1-3 m \autocite{sozzi2018benchmark}, a threshold easily met by consumer-grade drones.

Rather than producing information-sparse snapshots, UAV-based systems generate dense spatial fields of observation, allowing detection of yield gradients, stress hotspots, and management inconsistencies. Studies have demonstrated their utility, in conjunction with widespread connectivity and fast data processing and analysis pipelines, in yield prediction \autocite{wang2023drone}, disease assessment \autocite{bai2023rice,guo2021uas}, and real-time field diagnostics \autocite{tripicchio2015towards}. Such use can be accomplished with an inexpensive, off-the-shelf, consumer-grade drone with a standard RGB (red, green, and blue) camera or, if greater information density is desired, multispectral imaging \autocite{kosmala2016season} (Figure \ref{fig:drone-systems-image}).

\begin{figure}

{\centering
\includegraphics[width=0.50\linewidth]{./DJI_mavic3.png}
\includegraphics[width=0.50\linewidth]{./parrot_bluegrass.png}
}

\caption{A representative system for visible imaging (left: DJI Mavic 3; Image reproduced from: \url{https://www.cameralk.com/product/dji-mavic-3-fly-more-combo}) and multi-spectral imaging  (right: Parrot Bluegrass; Image reproduced from: \url{https://www.dronesdirect.co.uk/p/pf726300/parrot-bluegrass-fields-ag}) drone systems.}\label{fig:drone-systems-image}
\end{figure}

Incidentally, both of our adjoining neighbor countries have presented themselves in the forefront of commercial drone manufacturing and marketing \autocite{djiweb2023} and applications exploration. With recent surges in drone start-ups in India, the country is expected to dominate agriculture drone market with cheap and affordable hardware \autocite{pathak2020use,droneapplication2018}, few years from now.

\subsection{Use of public-domain data for crop suitability mapping}\label{use-of-public-domain-data-for-crop-suitability-mapping}

Although intensive crop-cut sampling provides reliable point estimates within the limits of its sampling design, its applicability depends on adherence to protocol and adequate sample coverage. In the digital era yield and cropping environments' suitability assessment can increasingly rely on integrated spatial inference, where soil, topography, climate, and vegetation indices jointly inform productivity potential.

The wealth of geographical and remotely sensed data publicly available, nowadays, can serve as a starting point in making informed decisions about scientific land management \autocite{huang2018agricultural}. With the focus of all three tiers of government on scientific land mapping for various purposes, identification of crop pockets will assist in agriculture production planning and mechanization. Farmholds and local planners benefit by having a concrete visual map of what focus is to be laid where. Moreover, this data driven information support system can help identify most profitable ventures, while tapping geographical comparative advantage.

\textcite{dhakal2024gridded} used publicly available (from NARC Nepal as raster layers) soil from 23,273 soils samples, collected from 56 districts covering seven provinces which were combined with a stack of 168 remote sensing-based soil covariates (SRTM DEM derivatives, climatic images, vegetation index etc.). Later the spatial predictions on 250x250m grids were generated using a machine learning method and the random forest. Thematic soil data on percentage Nitrogen content (PNC) and absolute pH value (pH) and elevation (EL) data layers were acquired for processing, supplemented with satellite based land use data (obtained from openly sourced OpenStreetMaps database) representing cultivated, residential and water bodies Figure \ref{fig:crop-map-tikapur} crop suitability map of multiple crops (as indicated in legend entries) in each of 58 grid pockets of Tikapur, Kailali, Nepal were created.

\begin{figure}[H]

{\centering \includegraphics[width=0.65\linewidth]{./tikapur_crop_suitability_simple_annotate_geometry} 

}

\caption{Crop suitability map showing depiction of crops with points scaled proportionate to their suitability rank.}\label{fig:crop-map-tikapur}
\end{figure}

Benefits from such approach is thanks to its ability to reduce dependence on subjective appraisal, enable transparent replication, support localized planning, scale naturally across administrative units. Furthermore, when combined with crop seasonality and agro-climatic variables, these maps become powerful decision-support tools for both farmers and planners.

\section{Way Forward}\label{way-forward}

Digital technologies offer Nepalese agriculture a rare opportunity to move beyond episodic intervention toward cumulative, learning-oriented transformation. Their inherent extensibility aligns well with Nepal's ecological diversity and institutional fluidity. However, realizing this potential requires more than hardware deployment. It demands robust system design, institutional commitment to continuous research, domain-specific digital literacy, and integration of digital tools with scientific and statistical rigor. Without these, digital agriculture risks becoming yet another cycle of enthusiasm and abandonment.

\printbibliography[title=Bibliography]

\section{Authors Contribution}\label{authors-contribution}

The manuscript was prepared exclusively by the named authors. The first author made the primary contribution to the development of the paper, including the selection of the research topic, comprehensive engagement with the contemporary literature, formulation of the review framework, articulation and discussion of the central concepts, development of visual materials, and synthesis of the final report. The second author contributed by proofreading the manuscript, revising minor sections, and enhancing its stylistic quality.

\section{Funding Information}\label{funding-information}

No financial support of any form, cash or kind, were either sought or utilized for research planning, scoping, executing and writing.

\section{Ethical Statement}\label{ethical-statement}

This manuscript is a systematic review of existing published research and does not involve new data collection from human or animal participants. Therefore, specific ethics committee approval was not sought. All literary sources are properly cited, and the work is an original analysis.

Figures and tables were created either with primarily collected data or secondary data, which when used has been properly credited. Semi-commercial or openly sourced media (image, graphs, videos, news) when reproduced are attributed as such.

\section{Acknowledgement}\label{acknowledgement}

Authors are grateful to their respective institutions of affiliation for their constant encouragement, and to their families for members' unconditional support and patience.

\end{document}
